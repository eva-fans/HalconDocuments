\documentclass{article}

\usepackage{xeCJK}

\begin{document}

\section{create\_shape\_model}
准备用于匹配的形状模型算子 create\_shape\_model 会将图像 Template 中传递的模板准备为用于匹配的形状模型。模型的 ROI 将作为 Template 的域传递。输出参数 ModelID 是该模型的句柄,用于后续的 find\_shape\_model 调用。模型图像模板域(区域)的重心被用作模型的原点(参考点)。可以使用 set\_shape\_model\_origin 设置不同的原点。模型通过多级图像金字塔生成并存储在内存中。如果选择了完整的模型预生成,则会在每个层次上以多次旋转的方式生成模型。可以使用 set\_shape\_model\_clutter 方法通过杂波参数扩展模型。

\section{create\_shape\_model\_xld}
根据 XLD 等值线准备匹配形状模型。

算子 create\_shape\_model\_xld 根据 Contours(轮廓)中传递的 XLD 轮廓创建用于匹配的形状模型。XLD 轮廓代表要搜索对象的灰度值边缘。与根据模板图像创建形状模型的算子 create\_shape\_model 相反,算子 create\_shape\_model\_xld 根据 XLD 轮廓创建形状模型,即不使用模板图像。

输出参数 ModelID 是该模型的句柄,在后续调用 find\_shape\_model 时会用到。轮廓图周围与坐标轴平行的最小矩形的重心被用作模型的原点(参考点)。可以使用 set\_shape\_model\_origin 设置不同的原点。模型为多级图像金字塔生成,并存储在内存中。如果选择完全预生成模型,则会在每个层次上生成多个旋转模型。可以使用 set\_shape\_model\_clutter 工具通过杂波参数对模型进行扩展。

\section{adapt\_shape\_model\_high\_noise}
针对高噪声图像调整形状模型参数。

adapt\_shape\_model\_high\_noise 针对高噪声图像调整形状模型 ModelID 的参数,并以 ResultDict 返回。根据样本搜索图像 ImageReduced,形状模型参数的最佳值将被估算并调整到模型中。ImageReduced 的域应包含模型和周围背景,以便向算法提供典型噪声。使用参数 GenParam 可以控制需要估算的参数。目前,运算器支持对匹配步骤中使用的最低金字塔级别进行估算。因此,应向 GenParam 提供"all"或"least"值。

\section{clear\_shape\_model}
释放形状匹配模型占用的内存。

算子 clear\_shape\_model 释放通过 create\_shape\_model、create\_scaled\_shape\_model 或 create\_aniso\_shape\_model 创建的形状模型的内存。调用 clear\_shape\_model 后,将无法再使用该模型。句柄 ModelID 将失效。

\section{create\_aniso\_shape\_model}
准备一个各向异性缩放的形状模型进行匹配。

算子 create\_aniso\_shape\_model 可将图像模板中传递的模板制作成用于匹配的各向异性缩放形状模型。模型的 ROI 将作为 Template 的域传递。输出参数 ModelID 是该模型的句柄,在后续调用 find\_aniso\_shape\_model 时使用。模型图像模板域(区域)的重心被用作模型的原点(参考点)。可以使用 set\_shape\_model\_origin 设置不同的原点。模型通过多级图像金字塔生成并存储在内存中。如果选择了完整的模型预生成,则会在每个层次上以多个旋转和各向异性比例(即行和列方向上的独立比例)生成模型。可以使用 set\_shape\_model\_clutter 方法通过杂波参数扩展模型。

\subsection{完全预生成模型}
可以选择在 Optimization 中传递第二个值。该值决定模型是否完全预生成。为此,优化的第二个值必须设置为 "预生成 "或 "无预生成"。如果没有使用第二个值(即只传递了一个值),则使用 set\_system('pregenerate\_shape\_models',...)设置的模式。如果使用默认值('pregenerate\_shape\_models' = 'false'),则不会完全预生成模型。

通常情况下,由于模型被完整预生成,模型无需在运行时进行转换,因此运行时间会稍短一些。不过,在这种情况下,内存需求和创建模型所需的时间都会大大增加。还应该注意的是,不能期望这两种模式返回完全相同的结果,因为在运行时转换模型必然导致转换后模型的内部数据与预先生成转换后模型的内部数据不同。例如,如果模型不是完全预生成的,find\_aniso\_shape\_model 通常会返回稍低的分数,这可能需要设置比完全预生成模型稍低的 MinScore 值。此外,在这两种模式下,通过插值获得的姿势可能会略有不同。如果希望获得最高精度,则应通过最小二乘调整来确定模型的姿态。

如果选择完全预生成模型,则会根据所选角度和比例范围预生成模型并存储在内存中。存储模型所需的内存与角度步数、比例步数和模型中的点数成正比。因此,如果AngleStep、ScaleRStep、ScaleCStep太小,或者AngleExtent、Scale Range太大,可能会导致模型无法再放入虚拟内存。在这种情况下,必须AngleStep、ScaleRStep或ScaleCStep,或者AngleExtent。在任何情况下,都希望模型完全适合主内存,因为这样可以避免操作系统分页,从而大大缩短查找对象的时间。由于可以通过 find\_aniso\_shape\_model 以亚像素分辨率确定角度,因此对于直径小于约 200 像素的模型,可以选择 AngleStep ≥ 1 和 ScaleRStep、ScaleCStep ≥ 0.02。

如果选择了 AngleStep = ′auto′ 或 ScaleRStep、ScaleCStep = ′auto′,则 create\_aniso\_shape\_model 会根据模型的大小自动分别确定合适的角度或比例步长。可以使用 get\_shape\_model\_params 查询自动计算出的角度和比例步长。

如果没有选择完全预生成模型,则只能在每个金字塔层以参考姿态创建模型。在这种情况下,模型必须在 find\_aniso\_shape\_model 中根据不同的角度和比例进行转换。因此,识别模型可能需要稍多的时间。

请注意,预生成的形状模型是根据特定图像尺寸定制的。由于运行时的原因,在使用同一模型并行搜索时,不支持使用不同尺寸的图像。在这种情况下,必须使用相同模型的副本,否则程序可能会崩溃!

\subsection{返回值}
如果参数有效,算子 create\_aniso\_shape\_model 返回值 2 (H\_MSG\_TRUE)。如有必要,则会出现异常。如果参数 "NumLevels "和 "Contrast "的选择导致模型包含的点数太少,则会出现错误 8510。

\section{create\_aniso\_shape\_model\_xld}
准备一个各向异性缩放的形状模型,以便根据 XLD 等值线进行匹配。

算子 create\_aniso\_shape\_model\_xld 根据 Contours(轮廓)中传递的 XLD 轮廓创建用于匹配的各向异性缩放形状模型。XLD 轮廓代表要搜索对象的灰度值边缘。与根据模板图像创建形状模型的算子 create\_aniso\_shape\_model 不同,算子 create\_aniso\_shape\_model\_xld 是根据 XLD 轮廓创建形状模型,即不使用模板图像。

输出参数 ModelID 是该模型的句柄,在后续调用 find\_aniso\_shape\_model 时使用。轮廓图周围与坐标轴平行的最小矩形的重心被用作模型的原点(参考点)。可以使用 set\_shape\_model\_origin 设置不同的原点。模型为多级图像金字塔生成,并存储在内存中。如果选择了完整的模型预生成,则会在每个层次上以多个旋转和各向异性比例(即行和列方向上的独立比例)生成模型。可以使用 set\_shape\_model\_clutter 工具通过杂波参数对模型进行扩展。

\section{create\_generic\_shape\_model}
算子 create\_generic\_shape\_model 会创建一个形状模型,并在 ModelID 中返回其句柄。

\section{create\_scaled\_shape\_model}
准备一个各向同性缩放的形状模型进行匹配。

算子 create\_scaled\_shape\_model 可以创建一个模板,通过图像 Template 传递给算子 create\_scaled\_shape\_model,作为用于匹配的各向同性缩放形状模型。模型的 ROI 将作为 Template 的域传递。

输出参数 ModelID 是该模型的句柄,会在后续调用 find\_scaled\_shape\_model 时使用。模型图像模板的域(区域)重心被用作模型的原点(参考点)。可以使用 set\_shape\_model\_origin 设置不同的原点。模型通过多级图像金字塔生成并存储在内存中。如果选择了完整的模型预生成,则会在每个层级上按多个旋转和比例生成模型。可以使用 set\_shape\_model\_clutter 方法通过杂波参数扩展模型。

\section{create\_scaled\_shape\_model\_xld}
准备一个各向同性缩放的形状模型,以便根据 XLD 等值线进行匹配。

算子 create\_scaled\_shape\_model\_xld 根据 Contours(轮廓)中传递的 XLD 轮廓创建用于匹配的各向同性缩放形状模型。XLD 轮廓代表要搜索对象的灰度值边缘。与根据模板图像创建形状模型的算子 create\_scaled\_shape\_model 不同,算子 create\_scaled\_shape\_model\_xld 是根据 XLD 轮廓创建形状模型,即不使用模板图像。

输出参数 ModelID 是该模型的句柄,在后续调用 find\_scaled\_shape\_model 时会用到。轮廓图周围与坐标轴平行的最小矩形的重心被用作模型的原点(参考点)。可以使用 set\_shape\_model\_origin 设置不同的原点。模型为多级图像金字塔生成,并存储在内存中。如果选择完全预生成模型,则会在每个层级上以多种旋转和比例生成模型。可以使用 set\_shape\_model\_clutter 方法通过杂波参数扩展模型。

\section{deserialize\_shape\_model}
反序列化序列化形状模型。

deserialize\_shape\_model 用于反序列化已被序列化的形状模型(有关序列化基本原理的介绍,请参阅 fwrite\_serialized\_item)。序列化后的形状模型由 SerializedItemHandle 句柄定义。反序列化后的值存储在一个自动创建的形状模型中,其句柄为 ModelID。

\section{determine\_shape\_model\_params}
确定形状模型的参数。

determine\_shape\_model\_params 可以根据模型图像模板自动确定形状模型的某些参数。要确定的参数可以用参数(Parameters)来指定。当创建形状模型(create\_shape\_model)、创建缩放形状模型(create\_scaled\_shape\_model)或创建形状模型(create\_aniso\_shape\_model)中的相应参数设置为 "auto"(自动)时,可以使用 determine\_shape\_model\_params 来确定自动确定的相同参数:金字塔的层数(参数 = 'num\_levels')、角度步长(参数 = 'angle\_step')、比例步长(各向同性缩放的参数 = 'scale\_step',各向异性缩放的参数 = 'scale\_r\_step' 和/或 'scale\_c\_step')、优化类型(参数 = 'optimization')、对比度阈值(参数 = 'contrast')或滞后阈值(参数 = 'contrast\_hyst')、对象部分的最小尺寸(参数 = 'min\_size')和最小对比度(参数 = 'min\_contrast')。通过在参数中传递上述值的元组,可以确定这些参数的任意组合。如果需要确定上述所有参数,则可以传递值 "all"。在这种情况下,两个迟滞阈值都会被确定,也就是说,算子的行为就像传递 "contrast\_hyst "而不是 "contrast"。

determine\_shape\_model\_params 主要用于在创建模型之前确定上述参数,例如,在交互式系统中,该系统会向用户提供有关这些参数的建议,但用户可以修改建议值。

自动确定的参数会以参数名(ParameterName)和参数值(ParameterValue)中的名-值对形式返回。ParameterName 中的参数名与 Parameters 中的参数名完全相同,当然,其中的值 "all "会被实际参数名替换。参数 "contrast\_hyst "是个例外,它会返回两个值 "contrast\_low "和 "contrast\_high"。

剩余的参数(NumLevels、AngleStart、AngleExtent、ScaleMin、ScaleMax、Optimization、Metric、Contrast和MinContrast)与create\_shape\_model、create\_scaled\_shape\_model和create\_aniso\_shape\_model中的含义相同。这些参数的描述可以在相应的算子文档中查找。这些参数将被determine\_shape\_model\_params操作使用,以与create\_shape\_model、create\_scaled\_shape\_model和create\_aniso\_shape\_model相同的方式计算要确定的参数。值得注意的是,如果要确定具有各向同性缩放的形状模型的参数,即如果Parameters中包含'scale\_step'(显式或隐式通过'all'表示),则ScaleMin和ScaleMax必须各自包含一个值。如果要确定具有各向异性缩放的形状模型的参数,即如果Parameters中包含'scale\_r\_step'或'scale\_c\_step'(显式或隐式表示),则ScaleMin和ScaleMax必须各自包含两个值。在这种情况下,各个参数的第一个值表示行方向的缩放,而第二个值表示列方向的缩放。

请注意,在determine\_shape\_model\_params中有一些参数可能也可以自动确定(NumLevels、Optimization、Contrast、MinContrast)。如果这些参数不想自动确定,即在Parameters中没有传递它们的名称,那么相应的参数必须包含有效的值,且不能设置为 "auto"。相反,如果这些参数要自动确定,它们的值将按照以下方式处理:如果优化或(滞后)对比度需要自动确定,即Parameters包含值"optimization"或"contrast"("contrast\_hyst"),则忽略Optimization和Contrast中传递的值。另一方面,如果要自动确定金字塔的最大层数或最小对比度,即Parameters包含值"num\_levels"或"min\_contrast",您可以让HALCON确定合适的值,并同时指定相应的上限或下限:

如果要预先指定金字塔的最大层数,可以将NumLevels设置为特定的值。如果在这种情况下,Parameters包含值"num\_levels",计算得到的金字塔层数最多为NumLevels。如果将NumLevels设置为"auto"(或0,以实现向后兼容性),则金字塔层数将在没有限制的情况下尽可能大。

如果要预先指定最小对比度,可以将MinContrast设置为特定的值。如果在这种情况下,Parameters包含值"min\_contrast",计算得到的最小对比度至少为MinContrast。如果将MinContrast设置为"auto",则最小对比度将在没有限制的情况下确定。

\section{find\_aniso\_shape\_model}
在图像中找到各向异性缩放的形状模型的最佳匹配。

算子find\_aniso\_shape\_model在输入图像Image中找到最佳的NumMatches个各向异性缩放形状模型ModelID实例。该模型必须先通过调用create\_aniso\_shape\_model或read\_shape\_model进行创建。找到的模型实例的位置、旋转和行列方向上的缩放分别以Row、Column、Angle、ScaleR和ScaleC返回。此外,每个找到的实例的匹配得分会在Score中返回。

\section{find\_aniso\_shape\_models}
寻找多个各向异性缩放形状模型的最佳匹配。

算子find\_aniso\_shape\_models在输入图像Image中找到多个各向异性缩放的形状模型,这些模型通过ModelIDs传递。这些模型必须先通过调用create\_aniso\_shape\_model或read\_shape\_model进行创建。与find\_aniso\_shape\_model不同的是,可以在一次调用中在同一图像中搜索多个模型。找到的模型实例的位置、旋转和行列方向上的缩放分别以Row、Column、Angle、ScaleR和ScaleC返回。每个找到的实例的匹配得分会在Score中返回。模型实例的类型会在Model中返回。

\section{find\_generic\_shape\_model}
在图像中找到一个或多个形状模型的最佳匹配。

算子find\_generic\_shape\_model在输入图像SearchImage中找到一个或多个形状模型,这些模型通过ModelID传递。找到的匹配结果会以MatchResultID的形式返回,并且可以使用get\_generic\_shape\_model\_result和get\_generic\_shape\_model\_result\_object进行查询。找到的匹配数量会在NumMatchResult中返回。可以使用set\_generic\_shape\_model\_param和set\_generic\_shape\_model\_object对模型进行参数化,从而可以控制搜索过程。

输入图像SearchImage的定义域设置了搜索的兴趣区域(ROI)。它限制了搜索空间,因为被视为ModelID重心的边界,以加快匹配过程。有关如何向SearchImage传递图像元组的信息,请参阅下面的说明。此外,搜索空间也受SearchImage大小的限制。默认情况下,ModelID只在定义域内完全适合的点中搜索。因此,如果ModelID超出图像边界,它将无法找到。这适用于所使用图像金字塔的所有级别。

在罕见情况下,通常发生在人工图像中,如果ModelID的实例在图像金字塔的任何级别上接触到定义域的边界,那么它们可能无法在图像中找到。一般来说,如果ModelID到图像边界的距离小于2NumLevels-1像素(其中NumLevels是金字塔级别的数量),则可能无法找到ModelID。可以通过使用"border\_shape\_models"参数来更改这种行为,参见set\_generic\_shape\_model\_param函数。

当搜索多个模型时,可以通过在SearchImage中传递具有缩小定义域的单个图像来同时为所有模型限制搜索空间。

另外,还可以通过传递一个包含多个图像对象的对象来为每个模型单独限制搜索空间。其中,对于ModelID中的每个模型,都会有一个相应的图像对象,这些图像对象一起组成了一个对象。这些图像对象除了定义域(domain)以外,其他属性必须完全相同,即图像对象的指针需要引用相同的图像(因此它们需要是完全相同的)。可以使用get\_image\_pointer1函数来检查这些图像对象的指针是否相同。

当应用多个模型时,必须为每个模型设置不同的标识符('model\_identifier'),以便每个实例可以与其找到的形状模型进行匹配。否则,会引发异常。

\section{find\_scaled\_shape\_model}
在图像中找到各向同性缩放的形状模型的最佳匹配。

算子find\_scaled\_shape\_model在输入图像Image中找到各向同性缩放的形状模型ModelID的最佳NumMatches个实例。该模型必须先通过调用create\_scaled\_shape\_model或read\_shape\_model进行创建。找到的模型实例的位置、旋转和缩放信息分别以Row、Column、Angle和Scale返回。每个找到的实例的匹配得分会在Score中返回。

\section{get\_generic\_shape\_model\_object}

\subsection{描述}
获取形状模型的标志性对象。

算子get\_generic\_shape\_model\_object读取形状模型ModelID的对象,并将它们以Object的形式返回。GenParamName指定将返回哪种标志性对象。以下值是可能的:

\begin{itemize}
	\item 'clutter\_region':形状模型的杂散区域。
	\item 'contours':表示形状模型ModelID在最底层金字塔级别上的XLD轮廓。因此,该轮廓被平移,使得具有坐标'origin\_row'和'origin\_column'(参见set\_generic\_shape\_model\_param)的模型的原点位于以像素为中心的HALCON标准坐标系的原点位置(参见Transformations / 2D Transformations)。例如,这些轮廓可以用于可视化在图像中找到的模型实例。
\end{itemize}

\subsection{返回值}
如果参数有效,算子get\_generic\_shape\_model\_object将返回值2(H\_MSG\_TRUE)。如果必要,将引发异常。

\section{get\_generic\_shape\_model\_param}
返回形状模型的参数。

算子get\_generic\_shape\_model\_param返回形状模型ModelID的GenParamName参数值GenParamValue。关于可设置的GenParamValue的解释,请参阅set\_generic\_shape\_model\_param。还有一些附加参数,其值只能被检索:
\begin{itemize}
	\item 'needs\_training':指示模型是否必须在使用find\_generic\_shape\_model之前通过train\_generic\_shape\_model进行训练('true')或否('false')。
	\item 'scale\_type':描述形状模型的缩放类型的字符串。
\end{itemize}

算子get\_generic\_shape\_model\_param返回形状模型使用的参数值。由于在调用train\_generic\_shape\_model后某些参数可能会被修改,它们的值可能与设置的值不同。这尤其适用于那些需要自动确定值的参数,例如,其值设置为'auto'。可以通过在其名称后添加后缀'\_param'来检索GenParamName的(未修改的)设置值。


\section{get\_generic\_shape\_model\_result}
算子get\_generic\_shape\_model\_result从形状匹配结果中获取字母数字值,并将其以GenParamValue形式返回。匹配结果包含在MatchResultID中,并按照得分降序排列。

参数MatchSelector用于选择从哪些匹配中返回结果值。排序将保持根据它们的得分进行。因此,它与传递给MatchSelector的选择条件的顺序无关。此外,即使满足多个选择条件,匹配结果也只返回一次。MatchSelector处理以下类型或它们的组合的选择条件:

\begin{itemize}
	\item 'all':选择所有匹配结果。
	\item 'best':选择得分最高的匹配。
\end{itemize}

结果索引:通过将MatchSelector设置为整数,确定在MatchResultID中匹配的索引。值范围:0到n-1(其中n为总结果数)。如果索引超出有效范围,将引发异常。
形状模型标识符:选择使用特定形状模型找到的所有匹配。为此,可以将相应的模型标识符'model\_identifier'作为字符串传递给MatchSelector。
形状模型句柄:选择使用特定形状模型找到的所有匹配。为此,可以将相应的形状模型句柄传递给MatchSelector。如果MatchSelector不是有效的选择条件,将引发异常。

GenParamName指定将返回哪种字母数字结果值。以下值是可能的:
\begin{itemize}

	\item 'num\_match\_result':匹配数量。在MatchSelector中进行多个选择时,返回匹配累积数量。
	\item 'model\_identifier':找到匹配的模型标识字符串。

	\item 'score':找到匹配的得分。它是介于0和1之间的数字,是模型在图像中被找到的好坏的近似度量。有关得分的更多信息,请参见“解决方案指南II-B - 匹配”,章节'General Topics',部分'About the Score'。

	\item 'angle':找到匹配的旋转角度。值范围:-3.14 ≤ 'angle' ≤ 3.14(=π)。

	\item 'row':在搜索图像中找到匹配的行坐标。该坐标与形状模型的原点相关。

	\item 'column':在搜索图像中找到匹配的列坐标。该坐标与形状模型的原点相关。

	\item 'scale\_row':在行方向上找到匹配的缩放。

	\item 'scale\_column':在列方向上找到匹配的缩放。

	\item 'hom\_mat\_2d':用于将形状模型的轮廓从其原点变换到搜索图像中找到的位置的单应性变换矩阵。考虑了位置、旋转和行列方向的缩放。有关使用单应性矩阵进行变换的更多信息,请参见affine\_trans\_point\_2d。

	\item 'clutter\_score':找到匹配的杂散得分。它是介于0和1之间的数字,是杂散区域中存在多少杂散边缘的近似度量。

	\item 'clutter\_hom\_mat\_2d':用于将形状模型的杂散区域从其原点变换到搜索图像中找到的位置的单应性变换矩阵。考虑了位置、旋转和行列方向的缩放。

	\item 'match':包含找到匹配的坐标'row'、'column',旋转角度'angle',缩放'scale\_row'、'scale\_column',得分'score'和单应性变换'hom\_mat\_2d'的字典。如果已经训练和使用了杂散区域(请参阅set\_generic\_shape\_model\_object和'clutter\_region'),字典中还包含两个特定于杂散区域的结果'clutter\_score'和'clutter\_hom\_mat\_2d'。

	\item 'time\_measurement':包含在搜索过程中进行的不同时间测量的字典。注意,它们的总和小于算子调用的总时间。该字典包含搜索流程的四个主要步骤的时间测量'pipeline':– 'pyramid':构建图像金字塔所花费的时间。– 'top\_level':在最高金字塔级别上搜索候选者所花费的时间。– 'tracking':从最高金字塔级别到最低级别跟踪候选者所花费的时间。– 'subpixel\_refinement':在搜索结束时进行细化所花费的时间。限制:MatchSelector必须设置为'all'。

\end{itemize}

请注意,对于涉及匹配位置的值,坐标给出为边缘中心坐标。

\section{get\_generic\_shape\_model\_result\_object}
算子get\_generic\_shape\_model\_result\_object从形状匹配结果中获取图标式结果,并将其以Objects形式返回。匹配结果包含在MatchResultID中。参数MatchSelector用于选择从哪些匹配中返回结果值。有关更多信息和MatchSelector支持的值列表,请参阅get\_generic\_shape\_model\_result。GenParamName指定将返回哪种对象结果值。以下值是可能的:
\begin{itemize}
	\item 'contours':根据匹配结果变换的模型轮廓。

	\item 'clutter\_region':不应出现杂散的区域。

\end{itemize}

\section{get\_shape\_model\_clutter}
可以使用算子get\_shape\_model\_clutter来检查之前使用set\_shape\_model\_clutter设置的形状模型ModelID的杂散参数。此外,还可以通过将GenParamName设置为'use\_clutter'来查询使用set\_shape\_model\_param设置的'use\_clutter'的值。

\section{get\_shape\_model\_origin}
算子get\_shape\_model\_origin返回形状模型ModelID的原点(参考点)。原点相对于用于使用create\_shape\_model、create\_scaled\_shape\_model或create\_aniso\_shape\_model创建形状模型的图像的区域的重心指定。因此,原点为(0,0)意味着使用模型图像的区域的重心作为原点。原点为(-20,-40)意味着原点位于重心的左上方。

\section{get\_shape\_model\_params}
算子get\_shape\_model\_params返回用于使用create\_shape\_model、create\_scaled\_shape\_model或create\_aniso\_shape\_model创建形状模型ModelID的参数。特别地,该输出可以用于检查参数NumLevels、AngleStep、ScaleStep和MinContrast,如果它们在使用create\_shape\_model、create\_scaled\_shape\_model或create\_aniso\_shape\_model在模型创建过程中被自动确定。如果使用adapt\_shape\_model\_high\_noise对ModelID进行了调整,则NumLevels包含两个值:最高金字塔级别和估计的最低金字塔级别。如果使用create\_shape\_model或create\_scaled\_shape\_model创建了形状模型,则ScaleMin、ScaleMax和ScaleStep中返回单个值。该参数对应于形状模型的各向同性缩放参数。如果使用create\_aniso\_shape\_model创建形状模型,则以上三个参数中每个都返回两个值。在这种情况下,各个参数的第一个值是行方向的缩放,而第二个值是列方向的缩放。请注意,参数Optimization和Contrast也可以在模型创建过程中自动确定,但无法通过使用get\_shape\_model\_params查询。

\section{inspect\_shape\_model}
inspect\_shape\_model创建一个形状模型的表示。该算子特别有用,因为它可以快速方便地确定在create\_shape\_model、create\_scaled\_shape\_model或create\_aniso\_shape\_model中使用的参数NumLevels和Contrast。模型的表示是在多个图像金字塔级别上创建的,级别的数量由NumLevels确定。与create\_shape\_model、create\_scaled\_shape\_model和create\_aniso\_shape\_model不同,该模型仅在输入图像中对象的旋转和缩放(即0°和1)上创建。作为输出,inspect\_shape\_model创建一个图像对象ModelImages,其中包含图像金字塔的各个级别的图像,以及一个ModelRegions中的区域,表示各个金字塔级别上的模型。可以使用select\_obj访问各个对象。

如果输入图像Image有一个通道,则模型的表示是使用create\_shape\_model、create\_scaled\_shape\_model或create\_aniso\_shape\_model中用于度量标准'use\_polarity'、'ignore\_global\_polarity'和'ignore\_local\_polarity'的方法创建的。如果输入图像具有多个通道,则表示是使用度量标准'ignore\_color\_polarity'的方法创建的。

与create\_shape\_model、create\_scaled\_shape\_model和create\_aniso\_shape\_model中的描述一样,应尽可能选择较大的金字塔级别数,同时要考虑在最高金字塔级别上必须能够识别模型并具有足够的模型点。应该选择适当的Contrast,以便仅使用模板对象的显著特征来创建模型。Contrast还可以包含具有两个值的元组。在这种情况下,将使用类似于边缘图像中使用的hysteresis阈值方法的方法对模型进行分割。元组的第一个元素确定较低的阈值,而第二个元素确定较高的阈值。有关hysteresis阈值方法的更多信息,请参见hysteresis\_threshold。

可选地,Contrast可以包含元组的第三个值作为元组的最后一个元素。该值根据组件的大小确定了选择显著模型组件的阈值,即小于指定最小大小的组件将被抑制。由于最小大小应用于组件的范围,因此导出的模型轮廓仍然可能小于指定的最小大小。对于每个连续的金字塔级别,此最小大小的阈值除以2。如果应该抑制小的模型组件,但仍然不执行hysteresis阈值处理,则Contrast必须指定三个值。在这种情况下,前两个值可以简单地设置为相同的值。

在典型的用法中,可以多次以不同的NumLevels和Contrast参数调用inspect\_shape\_model,直到获得满意的模型为止。然后使用获得的参数调用create\_shape\_model、create\_scaled\_shape\_model或create\_aniso\_shape\_model。

\section{create\_component\_model}
create\_component\_model用于准备基于显式指定的组件和关系的模板匹配。此算子接受模型图像ModelImage和多个区域ComponentRegions的形式,作为用于匹配的组件模型的组件模式。输出参数ComponentModelID是此模型的句柄,将在后续调用find\_component\_model时使用。与create\_trained\_component\_model不同,在调用create\_component\_model之前不需要进行train\_model\_components的训练。ComponentRegions中传递的每个区域描述一个单独的模型组件。稍后,使用ComponentRegions中的组件区域的索引来表示模型组件。每个组件的参考点是与其相关区域的重心,它在ComponentRegions中传递。可以通过调用area\_center来计算参考点。模型组件的相对移动(关系)可以通过传递VariationRow、VariationColumn和VariationAngle来设置。由于直接传递关系比较复杂,因此不使用关系,而是传递模型组件的变化。变化描述了组件相对于模型图像中的位置独立地进行的移动。参数VariationRow和VariationColumn分别通过±1/2VariationRow和±1/2VariationColumn来描述组件在行方向和列方向上的移动。参数VariationAngle通过±1/2*VariationAngle来描述组件的角度变化。基于这些值,自动计算关系。三个参数必须包含一个元素,这种情况下,参数用于所有模型组件,或者必须包含与ComponentRegions中的区域数量相同的元素数,这种情况下,每个参数元素都与ComponentRegions中的相应模型组件相关联。

参数AngleStart和AngleExtent确定模板模型在图像中可能的旋转范围。在内部,为每个模型组件构建了一个单独的形状模型(参见create\_shape\_model)。因此,参数ContrastLowComp、ContrastHighComp、MinSizeComp、MinContrastComp、MinScoreComp、NumLevelsComp、AngleStepComp、OptimizationComp、MetricComp和PregenerationComp对应于create\_shape\_model的参数,有以下差异:

首先,在create\_shape\_model的参数Contrast中,可以传递hysteresis阈值方法的上下阈值。此外,还可以传递第三个值,该值抑制了小的连接轮廓区域。相反,算子create\_component\_model通过提供三个单独的参数ContrastHighComp、ContrastLowComp和MinScoreComp来设置这三个值。因此,对于对比度阈值的自动计算也不同。如果两个hysteresis阈值都应自动确定,则必须将ContrastLowComp和ContrastHighComp都设置为“auto”。相反,如果只需确定一个阈值值,则ContrastLowComp必须设置为“auto”,而ContrastHighComp必须设置为与“auto”不同的任意值。

其次,create\_shape\_model的参数Optimization提供了减少模型点数的可能性以及完全预生成形状模型的可能性。相反,算子create\_trained\_component\_model使用单独的参数PregenerationComp来决定是否应完全预生成形状模型。

第三,应注意有关参数MinScoreComp。在使用基于形状的匹配时,在使用create\_shape\_model创建模板模型时不需要传递此参数,而仅在使用find\_shape\_model进行搜索时需要传递。相反,在准备组件模型时,最好分析模型组件的旋转对称性和模型组件之间的相似性。但是,仅当在搜索过程中(见find\_component\_model)使用的MinScoreComp值已经大致知道时,此分析才会导致有意义的结果。

除了参数ContrastLowComp、ContrastHighComp和MinSizeComp之外,还可以通过将相应参数设置为“auto”来自动确定参数MinContrastComp、NumLevelsComp、AngleStepComp和OptimizationComp。所有特定于组件的输入参数(参数名称以Comp结尾)必须包含一个元素,这种情况下,参数用于所有模型组件,或者必须包含与ComponentRegions中的区域数量相同的元素数,这种情况下,每个参数元素都与ComponentRegions中的相应元素相关联。

为了实现组件模型中单个模型组件之间的搜索方式,组件被表示为树结构。首先,搜索树的根部件(根组件)被搜索。然后,其余组件相对于搜索树中其前身的姿态进行搜索。可以在搜索过程中将根组件作为find\_component\_model的输入参数传递。选择哪个模型组件作为根组件取决于几个因素。原则上,应该选择在图像中可以高概率找到的模型组件。因此,有时高度遮挡的组件或在某些情况下缺失的组件不适合作为根组件。另外,与根组件相关联的计算时间也可以作为一个标准。根组件的排序可以基于后一标准返回在RootRanking中。在此参数中,模型组件的索引按降序排序,以其关联搜索时间。例如,RootRanking[0]包含选择为根组件时允许最快搜索的模型组件的索引。

需要注意的是,RootRanking返回的排序只是一个粗略估计。此外,根排名的计算假设在调用create\_component\_model和find\_component\_model时图像大小和系统参数'border\_shape\_models'的值相同。

总结起来,create\_component\_model可以用于准备一个组件模型,其中组件和它们之间的关系都是显式指定的。组件模型包含有关单个模型组件的信息以及它们之间搜索的方式。这种组件模型可以用于后续的find\_component\_model操作,从而在图像中进行形状匹配。

\section{create\_trained\_component\_model}
create\_trained\_component\_model用于准备基于训练组件的组件模型,其训练结果由ComponentTrainingID传递。输出参数ComponentModelID是此模型的句柄,将在后续调用find\_component\_model时使用。与create\_component\_model不同,在调用create\_trained\_component\_model之前必须先使用train\_model\_components训练模型组件。参数AngleStart和AngleExtent确定模板模型在图像中可能的旋转范围。在内部,为每个模型组件构建了一个单独的形状模型(参见create\_shape\_model)。因此,参数MinContrastComp、MinScoreComp、NumLevelsComp、AngleStepComp、OptimizationComp、MetricComp和PregenerationComp对应于create\_shape\_model的参数,有以下差异:

首先,create\_shape\_model的参数Optimization提供了减少模型点数的可能性以及完全预生成形状模型的可能性。相反,算子create\_trained\_component\_model使用单独的参数PregenerationComp来决定是否应完全预生成形状模型。

第二,关于参数MinScoreComp应该注意。在使用基于形状的匹配时,准备使用create\_shape\_model创建模板模型时不需要传递此参数,而仅在使用find\_shape\_model进行搜索时需要传递。相反,在准备组件模型时,最好分析模型组件的旋转对称性和模型组件之间的相似性。但是,只有在搜索过程中使用的MinScoreComp值已经大致知道的情况下,此分析才会导致有意义的结果。

在find\_component\_model的搜索后,返回搜索图像中组件的姿态参数。需要注意的是,姿态参数是相对于组件的参考点而言的。组件的参考点是与train\_model\_components返回的ModelComponents中的相关区域的重心。参数MinContrastComp、NumLevelsComp、AngleStepComp和OptimizationComp可以通过将相应参数设置为“auto”来自动确定。所有特定于组件的输入参数(参数名称以Comp结尾)必须包含一个元素,这种情况下,参数用于所有模型组件,或者必须包含与ComponentTrainingID中的模型组件数量相同的元素数,这种情况下,每个参数元素都与ComponentTrainingID中的相应组件相关联。

除了各个形状模型之外,组件模型还包含有关如何使用find\_component\_model在相互之间搜寻单个模型组件以最小化搜索计算时间的信息。为此,组件以树形结构表示。首先,搜索树的根组件(根组件)被搜索。然后,其余组件相对于搜索树中其前身的姿态进行搜索。在搜索过程中,根组件可以作为find\_component\_model的输入参数进行传递。选择哪个模型组件作为根组件取决于几个因素。原则上,应该选择在图像中可以高概率找到的模型组件。因此,有时高度遮挡的组件或在某些情况下缺失的组件不适合作为根组件。另外,与根组件相关联的计算时间也可以作为一个标准。根组件的排序可以基于后一标准返回在RootRanking中。在此参数中,模型组件的索引按降序排序,以其关联的搜索时间。例如,RootRanking[0]包含选择为根组件时允许最快搜索的模型组件的索引。

需要注意的是,RootRanking返回的排序只是一个粗略估计。此外,根排名的计算假设在调用create\_trained\_component\_model和find\_component\_model时图像大小和系统参数'border\_shape\_models'的值相同。

\section{find\_component\_model}
\subsection{描述}
在图像中找到组件模型的最佳匹配。算子 find\_component\_model 在输入图像 Image 中找到组件模型 ComponentModelID 的最佳匹配 NumMatches 个实例。搜索结果可以通过 get\_found\_component\_model 进行可视化。此算子还可用于提取特定组件模型实例的组件匹配。

\subsection{输入参数}
ComponentModelID: 组件模型的句柄。该模型必须先通过调用create\_trained\_component\_model或create\_component\_model进行创建,或者使用read\_component\_model进行读取。

RootComponent: 根组件的索引。组件模型ComponentModelID中的组件以树形结构表示。位于搜索树根部的组件即为根组件。根组件在全搜索空间中进行搜索,即在所有允许的位置和允许的取向范围内。而其余组件相对于其前导组件的姿态在限制的搜索空间中进行搜索。组件作为根组件的适用程度取决于多种因素。原则上,应选择在图像中以高概率找到的模型组件作为根组件。因此,在图像中有时被遮挡较多或有时缺失的组件不适合作为根组件。另一种可能的标准是在搜索过程中与根组件相关联的计算时间。基于该标准的模型组件排名将在算子create\_trained\_component\_model或create\_component\_model的RootRanking中返回。如果完整排名被传递给RootComponent,则将自动选择第一个值RootComponent[0]作为根组件。

AngleStartRoot和AngleExtentRoot: 指定根组件搜索的允许角度范围(单位为弧度)。如果需要,旋转范围会被剪裁为创建create\_trained\_component\_model或create\_component\_model时给定的范围。每个组件的角度范围可以通过get\_shape\_model\_params查询。对应的形状模型的句柄可以通过get\_component\_model\_params获得。

MinScore: 确定组件模型的潜在匹配至少需要具有的分数。如果在图像中可以预期组件模型永远不会被遮挡,则可以将MinScore设置得很高,如0.8或0.9。该参数的值对计算时间的影响较小。一个例外是当IfRootNotFound设置为'select\_new\_root'时。

NumMatches: 确定返回的实例最大数量。如果找到的实例少于NumMatches,则仅返回该数量,即参数MinScore优先于NumMatches。如果在图像中找到了多于NumMatches个分数大于MinScore的实例,则仅返回最佳的NumMatches个实例。然而,如果希望找到图像中所有分数超过MinScore的模型实例,则必须将NumMatches设置为0。

MaxOverlap: 确定两个实例之间最大允许重叠的部分,此值是一个介于0和1之间的数字。在某些情况下,找到的实例在一个或几个组件的姿态上有所不同。如果两个实例之间的重叠超过MaxOverlap,则仅返回最佳实例。这意味着对于MaxOverlap = 0,找到的实例不能有任何重叠,而对于MaxOverlap = 1,则不进行重叠检查,因此将返回所有实例。重叠的计算基于找到的组件实例的任意方向最小包围矩形。

IfRootNotFound: 指定算子处理缺失或严重遮挡根组件时的行为。可能的值:
\begin{itemize}
	\item ’stop\_search’:假定根组件总是在图像中找到。因此,对于无法找到根组件的实例,将不会继续搜索其他组件。
	\item ’select\_new\_root’:逐步选择不同的组件作为根组件,并在全搜索空间内进行搜索。选择根组件的顺序与RootRanking中传递的顺序相对应。然后,使用所有根组件的找到的实例的姿态开始递归搜索其他组件。因此,即使未找到原始根组件,仍有可能找到实例。然而,与选择'stop\_search'时相比,这会显著增加搜索的计算时间。特别是对于MinScore的小值,因为需要搜索更多的根组件。如果在搜索过程中需要比RootComponent中给定的更多的根组件,则自动使用计算得到的顺序来完成根组件。

\end{itemize}

IfComponentNotFound: 指定当前一个组件未找到时(例如,因为组件丢失或遮挡严重)如何搜索后续组件。可能的值:
\begin{itemize}
	\item ’prune\_branch’:这些组件根本不会被搜索,也会被视为“未找到”。

	\item ’search\_from\_upper’:这些组件相对于前导组件的前导组件的姿态进行搜索。

	\item ’search\_from\_best’:这些组件相对于已找到的组件的姿态进行搜索,从而可以以最少的计算工作量进行相对搜索。
\end{itemize}

PosePrediction: 确定是否应估计未找到组件的姿态。可能的值:
\begin{itemize}
	\item ’none’:仅返回找到的组件的姿态。

	\item ’from\_neighbors’:估计未找到组件的姿态,并以ScoreComp = 0.0的得分返回。姿态估计基于搜索树中找到的相邻组件的姿态。

	\item ’from\_all’:估计未找到组件的姿态,并以ScoreComp = 0.0的得分返回。姿态估计基于所有找到的组件的姿态。
\end{itemize}

MinScoreComp: 对于找到实例所需的组件最低得分。此参数与find\_shape\_model中的MinScore具有相同的含义。可设置为一个元素或与ComponentModelID中的模型组件数量相同的元素数目。在第一种情况下,该参数用于所有组件。在第二种情况下,每个参数元素与ComponentModelID中对应的组件相关联。

SubPixelComp: 确定是否以亚像素精度进行提取,并在该情况下允许的最大对象变形(以像素为单位)。此参数与find\_shape\_model中的SubPixel具有相同的含义。因此,最大允许的对象变形必须以整数形式给出在同一个字符串中。可设置为一个元素或与ComponentModelID中的模型组件数量相同的元素数目。在第一种情况下,该参数用于所有组件。在第二种情况下,每个参数元素与ComponentModelID中对应的组件相关联。示例:['least\_squares','max\_deformation 2']。

NumLevelsComp: 确定匹配中使用的组件的金字塔级别。此参数与find\_shape\_model中的NumLevels具有相同的含义。它决定要在匹配中使用的组件的金字塔级别数。可设置为一个元素或与ComponentModelID中的模型组件数量相同的元素数目。在第一种情况下,该参数用于所有组件。在第二种情况下,每个参数元素与ComponentModelID中对应的组件相关联。可选,可以为该参数设置值对:在这种情况下,第一个值仍然确定要使用的金字塔级别数。第二个值指定要追踪的找到的匹配的最低金字塔级别。因此,可以为每个模型组件在ComponentModelID中设置单个值对,如果要为不同组件使用不同的值对,则必须在同一个元组中指定它们。示例:ComponentModelID包含两个组件,其中要考虑不同的金字塔级别。对于第一个组件,要使用5个级别,直到级别2。对于第二个组件,要使用4个级别,直到级别1。在这种情况下,NumLevelsComp = [5,2,4,1]。

GreedinessComp: 启发式组件搜索的“贪婪程度”:取值范围为0到1。其中0表示“安全但慢”,1表示“快但可能会错过匹配”。此参数与find\_shape\_model中的Greediness具有相同的含义。可设置为一个元素或与ComponentModelID中的模型组件数量相同的元素数目。在第一种情况下,该参数用于所有组件。在第二种情况下,每个参数元素与ComponentModelID中对应的组件相关联。

\subsection{输出参数}
ModelStart和ModelEnd:返回与组件模型的同一实例相关联的所有组件匹配的第一个和最后一个索引,从而给出索引范围。与组件模型的第一个找到的实例对应的组件匹配由索引区间[ModelStart[0],ModelEnd[0]]给出。其中索引指的是参数RowComp,ColumnComp,AngleComp,ScoreComp和ModelComp的值。示例:组件模型由三个组件组成。在图像上找到了两个实例,其中一个实例只找到了两个组件(组件0和组件2)。然后返回的参数可能如下所示:RowComp = [100,200,300,150,250] ModelStart = [0,3] ColumnComp = [200,210,220,400,425] ModelEnd = [2,4] AngleComp = [0,0.1,-0.2,0.1,0.2] ModelComp = [0,1,2,0,2] ScoreComp = [1,1,1,1,1] Score = [1,1] 从右列可见,左列中索引0到2的值对应于实例1的组件0到2;索引3到4的值对应于实例2的组件0和2。

Score:组件模型找到的实例的得分。Score包含组件得分的加权平均值,即ScoreComp中的值。权重是根据各组件内模型点的数量进行的。

RowComp、ColumnComp和AngleComp:所有找到的组件模型实例的模型组件的位置(RowComp,ColumnComp)和旋转(AngleComp)。坐标RowComp和ColumnComp是搜索图像中组件原点(参考点)的坐标。组件原点取决于模型创建方式:

\begin{itemize}
  \item 通过训练create\_trained\_component\_model:组件原点是算子 train\_model\_components的ModelComponents中返回轮廓区域的重心。

  \item 通过手动创建create\_component\_model:组件原点是算子 create\_component\_model中传递组件区域ComponentRegions的重心。
\end{itemize}

由于ComponentModelID中的组件之间的关系是相对于这个参考点的,因此不应通过使用set\_shape\_model\_origin来修改组件的原点。

ScoreComp:找到的每个组件实例的得分。得分是一个介于0和1之间的数值,是组件在图像中可见程度的近似度量。例如,如果组件的一半被遮挡,得分不会超过0.5。

ModelComp:找到组件的索引。该元组包含相应模型组件的索引。通过这一点,可以将RowComp,ColumnComp,AngleComp和ScoreComp的值与不同的模型组件相关联。

\subsection{搜索相关信息}
在组件匹配中,内部使用基于形状的匹配来搜索单个组件。图像的域确定了参考点的搜索空间,即根组件的允许位置。通常情况下,组件模型仅在图像域的那些能够完整容纳模型的点中进行搜索。这意味着,即使组件的得分高于MinScoreComp,如果它们延伸到图像的边界之外,它们也将无法被找到。需要注意的是,如果对于某个金字塔级别,组件模型刚好触及图像边界,即使它完全位于原始图像内,也可能找不到。作为经验法则,如果模型到图像边界的距离低于2NumLevels−1,可能无法找到该模型。可以通过使用set_system('border_shape_models', 'true')来改变这种行为,这将导致如果组件延伸超出图像边界且其得分高于MinScoreComp,则它们会被找到。在这种情况下,位于图像之外的点被视为被遮挡的,即它们会降低得分。需要注意的是,这种模式会增加搜索的运行时间。此外,对于一些罕见的情况,通常仅发生在人工图像中,即使在某些金字塔级别上模型触及了减小域的边界,也可能无法找到模型。这时,可以通过使用dilation_circle等方法,将减小的域扩大2NumLevels−1个单位来解决。在通过图像金字塔跟踪匹配时,在每个级别上,根据NumMatches参数拒绝一些较不理想的匹配。因此,有可能拒绝了一些在最低金字塔级别上得分较高的匹配。因此,例如,当NumMatches设置为1时,找到的匹配可能与将NumMatches设置为0或> 1时返回的得分最高的匹配不同。

\subsection{推荐}
如果预期有多个得分相似的对象,但只想返回得分最高的那个,建议增加NumMatches的值,然后选择得分最高的匹配。为了获得有意义的得分值并避免错误匹配,我们建议始终将允许变形与应用最小二乘调整的亚像素提取相结合。如果为每个组件单独指定亚像素提取和/或最大对象变形值,则对于ComponentModelID中的每个组件,必须传递一个SubPixelComp值来指定亚像素提取。在每个亚像素提取值之后,可以选择传递一个第二个值,用于描述相应模式的最大对象变形值。如果某个组件没有传递最大对象变形值,将在搜索该组件时不考虑变形。

\section{gen\_initial\_components}
gen\_initial\_components函数用于提取组件模型的初始组件。一般有两种用法:第一种情况是在组件模型的组件不清楚的情况下,gen\_initial\_components会自动从模型图像中提取组件模型的初始组件。第二种情况是在组件模型的组件大致已知的情况下,可以使用gen\_initial\_components来为train\_model\_components和create\_component\_model中的模型特征提取找到合适的参数值。

当使用第一种情况时,gen\_initial\_components会从模型图像ModelImage中提取组件模型的初始组件。需要注意的是,这在组件模型的组件不清楚的情况下尤其有用。在这种情况下,得到的初始组件可以用于自动训练组件模型,train\_model\_components会提取(最终的)模型组件及其之间的关系。gen\_initial\_components会返回一个region对象元组InitialComponents,其中包含每个初始组件的表示形式,以轮廓区域的形式呈现。为了自动确定初始组件,模型图像ModelImage的域必须包含整个复合对象,包括所有组件。Mode参数用于指定用于自动计算的方法。目前,只有"connection"模式可用。在该模式下,自动计算分为两个步骤进行:首先,使用ContrastLow、ContrastHigh和MinSize参数提取特征。这三个参数定义了初始组件应该包含的轮廓,并且应该选择这样的参数值,以使模型图像中仅包含显著特征。ContrastLow和ContrastHigh指定应包含在初始组件中的点的灰度值对比度。对比度是对象与背景以及对象的不同部分之间的局部灰度值差异的度量。模型图像使用类似于边缘图像中使用的滞后阈值方法进行分割。其中,ContrastLow确定较低的阈值,而ContrastHigh确定较高的阈值。如果ContrastLow和ContrastHigh传递相同的值,则执行简单的阈值操作。有关滞后阈值方法的更多信息,请参阅hysteresis\_threshold函数。MinSize参数可用于根据连接轮廓区域的大小仅选择显著特征用于初始组件,即连接轮廓区域的点数少于MinSize的被抑制。在第二步中,迭代地合并生成的连接轮廓区域。为此,如果两个轮廓区域之间的距离小于某个阈值(见下文),则将两个轮廓区域合并。最后,合并后的区域将以InitialComponents的形式返回,并可以将其传递给train\_model\_components以训练组件模型。为了控制内部图像处理,使用了参数GenericName和GenericValue。通过在GenericName中作为字符串列表传递要更改的控制参数的名称,在GenericValue中传递相应的值。通常情况下,不需要更改任何值。只有在自动确定初始组件的结果不满意时才应进行更改。可以更改的两个参数是'merge\_distance'和'merge\_fraction',这两个参数用于迭代合并轮廓区域(见上文)。首先,计算一个轮廓区域的轮廓像素中最多与另一个轮廓区域距离为'merge\_distance'的分数。如果这个分数超过'merge\_fraction'参数传递的值,那么这两个轮廓区域将合并。因此,选择较大的'merge\_distance'和较小的'merge\_fraction'将合并更多的轮廓区域。'merge\_distance'的默认值为5,'merge\_fraction'的默认值为0.5(对应50\%)。当使用第二种情况,即组件模型的组件大致已知时,可以在不先执行gen\_initial\_components的情况下使用train\_model\_components进行训练。如果希望这样做,可以由用户指定初始组件,并直接将其传递给train\_model\_components。此外,如果组件及其相对运动(关系)已知,则无需执行gen\_initial\_components和train\_model\_components。实际上,通过立即将组件及其关系传递给create\_component\_model,可以在不进行任何训练的情况下创建组件模型。在这两种情况下,gen\_initial\_components仍然可以用于评估train\_model\_components和create\_component\_model中的特征提取参数ContrastLow、ContrastHigh和MinSize的效果,并找到适用于特定应用的参数值。在这种情况下,必须明确给出(初始)组件的图像区域,即必须传递每个(初始)组件应该创建的单独图像。在这种情况下,ModelImage包含多个图像对象。将每个图像对象的域用作计算相应(初始)组件的兴趣区域。元组中所有图像对象的图像矩阵必须相同,即ModelImage不能使用concat\_obj以任意方式构造,而必须使用add\_channels或等效调用从同一图像创建。如果不是这种情况,将返回错误消息。如果参数ContrastLow、ContrastHigh或MinSize只包含一个元素,则该值适用于所有(初始)组件的创建。相反,如果不同的值应该用于不同的(初始)组件,则可以为这三个参数传递值的元组。

在这种情况下,元组的长度必须与(初始)组件的数量相对应,即与ModelImage中的图像对象数量相同。生成的(初始)组件的轮廓区域将在InitialComponents中返回。

综上所述,第二种情况与基于形状匹配中的inspect\_shape\_model函数类似。然而,与inspect\_shape\_model不同,gen\_initial\_components不会返回多个图像金字塔级别上的轮廓区域。因此,如果需要手动选择要使用的金字塔级别数量,最好为每个(初始)组件单独调用inspect\_shape\_model。

在这两种情况下,参数ContrastLow、ContrastHigh和MinSize可以自动确定。如果两个滞后阈值都要自动确定,ContrastLow和ContrastHigh都必须设置为"auto"。相反,如果只想确定一个阈值值,ContrastLow必须设置为"auto",而ContrastHigh必须设置为与"auto"不同的任意值。

如果输入图像ModelImage具有一个通道,那么生成的模型表示采用create\_component\_model或create\_trained\_component\_model函数用于指标"use\_polarity"、"ignore\_global\_polarity"和"ignore\_local\_polarity"的方法。如果输入图像具有多个通道,则表示采用用于指标"ignore\_color\_polarity"的方法。

总结而言,gen\_initial\_components函数可用于两种情况:一是从模型图像中自动提取组件模型的初始组件;二是在已知组件的情况下,为train\_model\_components和create\_component\_model提供合适的参数值。该函数对于自动确定特征提取参数和创建组件模型都非常有用,可以根据具体应用场景灵活选择合适的用法。


\end{document}
